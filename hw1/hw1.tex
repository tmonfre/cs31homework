\documentclass[11pt]{article}
\setlength{\oddsidemargin}{0in}
\setlength{\evensidemargin}{0in}
\setlength{\textwidth}{6.5in}
\setlength{\topmargin}{0in}
\setlength{\textheight}{8.5in}
\setlength{\headheight}{0pt}
\setlength{\headsep}{0pt}

\setcounter{topnumber}{3}%
\def\topfraction{.7}% 
\setcounter{bottomnumber}{1}
\def\bottomfraction{.3}
\setcounter{totalnumber}{5}%
\def\textfraction{.1}% was .2
\def\floatpagefraction{.7}% was .7
\setcounter{dbltopnumber}{2}
\def\dbltopfraction{.7}
\def\dblfloatpagefraction{.5}

\usepackage{clrscode3e}
\usepackage[parfill]{parskip}
\usepackage{textcomp}
\usepackage[T1]{fontenc}
\usepackage{titling}
\usepackage[shortlabels]{enumitem}

% set title height
\setlength{\droptitle}{-4em}

% remove number from section
\makeatletter
\renewcommand{\@seccntformat}[1]{%
  \ifcsname prefix@#1\endcsname
    \csname prefix@#1\endcsname
  \else
    \csname the#1\endcsname\quad
  \fi}
\newcommand\prefix@section{}
\makeatother

% define how to make line breaks
\def\separateline{\medskip\hrule\medskip}

% define title
\title{CS 31: Homework 1}
\author{Thomas Monfre}
\date{\today}

\begin{document}
\maketitle

\section{Problem 1-1}
Prove the following. Let $A$ be an event and $X_A = I\{A\}$ be the indicator random variable for $A$. Then for all real numbers $c > 0$, we have $E[X_A^c] = Pr\{A\}$.
\separateline
\\See textbook page 118 and 119 for help on this. What does the superscript c mean?

\newpage

\section{Problem 1-2}
Exercise 2.3-7: Describe a $\Theta(n\lg{n})$-time algorithm that, given a set $S$ of $n$ integers and another integer $x$, determines whether or not there exists two elements in $S$ whose sum if exactly $x$.
\separateline
My answer here.

\newpage

\section{Problem 1-3}
In class, we saw a lower-bound argument showing that the worst-case running time of insertion sort is $\Omega(n^2)$. The argument was based on dividing the array of $n$ elements into three sections, each of size $\frac{n}{3}$.

Suppose that $\alpha$ is a fraction in the range $0 < \alpha < 1$. Show how to generalize the lower-bound argument for insertion sort to consider an input in which the $\alpha n$ largest values start in the first $\alpha n$ positions. What additional restriction do you need to put on $\alpha$? What value of $\alpha$ maximizes the number of times that the $\alpha n$ largest values must pass through the middle $1 - 2\alpha$ array positions?
\separateline
\textbf{Going to have to review my notes on this one.}

But for the generalization, I'm thinking that it will work so long as the middle section that we multiply by encompasses the whole thing.

Therefore, the middle section should be $1 - 2(\frac{n}{k})$

For the example of $\frac{n}{3}$, $\alpha = \frac{1}{3}$, so the middle section = $1 - 2(\frac{n}{3})$ = $1 - \frac{2n}{3}$ = $\frac{n}{3}$.

Then we just have to multiply $\frac{n}{k} * (1 - 2(\frac{n}{k})) = \Theta(n^2)$

Work on cleaning this up, but that is the idea

ADDITIONAL RESTRICTION is that it must be less than 0.5

\newpage

\section{Problem 1-4}
Exercise 4.2-4: What is the largest $k$ such that if you can multiply $3 \times 3$ matrices using $k$ multiplications (not assuming commutativity of multiplication), then you can multiply $n \times n$ matrics in time $o(n^\lg{7})$? What would the running time of this algorithm be?
\separateline

Read up on Strassen's Algorithm.

\newpage

\section{Problem 1-5}
Problem 4-1. Use the master method when applicable. Don’t worry about base cases.
\separateline

\begin{enumerate}[(a)]
\item $T(n) = 2T(n/2) + n^4$

Using the master method we get $a = 2$, $b = 2$, and $f(n) = n^4$. Therefore, we must compare $n^\log_2{2}$ with $n^4$.

Since $\log_2{2} = 1$, and $1 < 4$, $f(n) = n^4$ is polynomially larger than $n^\log_2{2}$. This indicates case 3 of the master method applies. Checking the regularity condition we see $\frac{1}{8}(n^4) \leq cn^4$ for some constant $c < 1$ and all sufficiently large $n$.

Therefore, we have \mathbf{$\Theta(n^4)$}.\\


\item $T(n) = T(7n/10) + n$


Using the master method we get $a = 1$, $b = \frac{10}{7}$, and $f(n) = n$. Therefore, we must compare $n^\log_\frac{10}{7} {1}$ with $n$.

Since $\log_\frac{10}{7} {1} = 0$, and $0 < 1$, $f(n) = n$ is polynomially larger than $n^\log_\frac{10}{7} {1}$. This indicates case 3 of the master method applies. Checking the regularity condition we see $\frac{7}{10}(n) \leq cn$ for some constant $c < 1$ and all sufficiently large $n$.

Therefore, we have \mathbf{$\Theta(n)$}.\\


\item $T(n) = 16T(n/4) + n^2$


Using the master method we get $a = 16$, $b = 4$, and $f(n) = n^2$. Therefore, we must compare $n^\log_4{16}$ with $n^2$.

Since $\log_4{16} = 2$, and $2 = 2$, $f(n) = n^2$ is within a polylog factor of $n^\log_4{16}$, but is not smaller. This indicates case 2 of the master method applies.

Therefore, we have \mathbf{$\Theta(n^2)$}.\\


\item $T(n) = 7T(n/3) + n^2$


Using the master method we get $a = 7$, $b = 3$, and $f(n) = n^2$. Therefore, we must compare $n^\log_3{7}$ with $n^2$.

Since $\log_3{7} = 1.7712437492$, and $1.7712437492 < 2$, $f(n) = n^2$ is polynomially larger than $n^\log_3{7}$. This indicates case 3 of the master method applies. Checking the regularity condition we see $\frac{7}{9}(n^2) \leq cn^2$ for some constant $c < 1$ and all sufficiently large $n$.

Therefore, we have \mathbf{$\Theta(n^2)$}.\\


\item $T(n) = 7T(n/2) + n^2$


Using the master method we get $a = 7$, $b = 2$, and $f(n) = n^2$. Therefore, we must compare $n^\log_2{7}$ with $n^2$.

Since $\log_2{7} = 2.8073549221$, and $2.8073549221 > 2$, $f(n) = n^2$ is polynomially smaller than $n^\log_2{7}$. This indicates case 1 of the master method applies.

Therefore, we have \mathbf{$\Theta(n^\log_2{7})$}.\\


\item $T(n) = 2T(n/4) + \sqrt{n}$


Using the master method we get $a = 2$, $b = 4$, and $f(n) = \sqrt{n} = n^\frac{1}{2}$. Therefore, we must compare $n^\log_4{2}$ with $n^\frac{1}{2}$.

Since $\log_4{2} = \frac{1}{2}$, and $\frac{1}{2} = \frac{1}{2}$, $f(n) = n^\frac{1}{2}$ is within a polylog factor of $n^\log_4{2}$, but is not smaller. This indicates case 2 of the master method applies.

Therefore, we have \mathbf{$\Theta(n^\frac{1}{2})$}.\\


\item $T(n) = T(n - 2) + n^2$


asdfasdfasdf
\end{enumerate}

\newpage

\section{Problem 1-6}
Problem 4-3, parts g and j. Don’t worry about base cases.
\separateline

\hspace*{6mm} (g)  $T(n) = T(n - 1) + \lg{n}$

\hspace*{12mm} asdfasdfasdf\\

\hspace*{6mm} (j)  $T(n) = \sqrt{n}$ \hspace $T(\sqrt{n}) + n$

\hspace*{12mm} asdfasdfasdf\\


 \newpage

 \section{Problem 1-7}
 Exercise 5.2-5: Let $A[1...n]$ be an array of $n$ distinct numbers. If $i < j$ and $A[i] > A[j]$, then the pair $(i,j)$ is called an \textbf{inversion} of $A$. Suppose that the elements of $A$ form a uniform random permutation of $\langle1,2,...,n\rangle$. Use indicator random variables to compute the expected number of inversions.
 \separateline

 asdfasdfasdfasdfafd


\end{document}
